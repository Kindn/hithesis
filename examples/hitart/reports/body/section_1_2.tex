\section{课题来源及研究的目的和意义}
\subsection{研究背景}
进入21世纪以来,多旋翼无人机领域取得了很大的发展,成为一类成功从实验室走进人们生活的机器人系统。
为突破现在市场上主流的欠驱动无人机所存在的瓶颈,近年来人们又研制出了不少种类冗余驱动多旋翼无人机系统。
作为一种全驱动系统,这种无人机可以跟踪6自由度轨迹,有效增强了多旋翼无人机的机动性能,拓宽了其应用场景。

作为一个高自由度且敏捷的系统,无人机的稳定自主飞行依赖一个鲁棒且高精度的导航系统来对自身和外界的状态进行有效估计。
同时定位与建图(Simultaneous Localization and Mapping, SLAM)技术被广泛应用于移动机器人的实时定位导航中,
使得无人机在复杂的无GPS环境中也能获得准确的状态估计和环境感知。
其中,基于视觉或视觉惯性融合的SLAM导航技术以其相对较低的成本、轻量级的硬件需求和可观的表现成为更适用与无人机的选择,
并且一直以来都是研究的热点。\cite{ding2021vid}
\subsection{研究的目的及意义}
\section{国内外研究现状及分析}
\subsection{国内外研究现状}
\subsubsection{视觉惯性导航研究现状}
\subsubsection{无人机的外力估计研究现状}
\subsubsection{全驱动旋翼无人机研究现状}
\subsection{国内外文献综述及简析}
\subsubsection{视觉惯性导航文献综述及简析}
\subsubsection{无人机的外力估计文献综述及简析}
\subsubsection{全驱动旋翼无人机文献综述及简析}

% Local Variables:
% TeX-master: "../report"
% TeX-engine: xetex
% End: